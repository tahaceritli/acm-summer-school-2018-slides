\documentclass{beamer}					% Document class

\usepackage[english]{babel}				% Set language
\usepackage[utf8x]{inputenc}			% Set encoding
\usepackage{tikz}
\usepackage{booktabs}
\usepackage{amsmath}
\usepackage{standalone}
\usepackage{dirtytalk}
\usepackage[numbers]{natbib}

\bibliographystyle{ieeetr}
\usetikzlibrary{arrows,automata}
%\usetikzlibrary{bayesnet}

\mode<presentation>						% Set options
{
  \usetheme{default}					% Set theme
  \usecolortheme{default} 				% Set colors
  \usefonttheme{default}  				% Set font theme
  \setbeamertemplate{caption}[numbered]	% Set caption to be numbered
}

 \setbeamertemplate{footline}[frame number]


\usepackage{graphicx}					% For including figures
\usepackage{booktabs}					% For table rules
\usepackage{hyperref}					% For cross-referencing

\title{Hands on Data Cleaning of Messy Data}	% Presentation title
\author{Taha Ceritli and Chris Williams}
\institute{School of Informatics, University of Edinburgh}
\date{\today}									% Today's date	

\newcommand{\cut}[1]{}
\begin{document}

\begin{frame}
  \titlepage
\end{frame}
   
%\begin{frame}
%\tableofcontents
%\end{frame}

 \begin{frame}[c]
  \frametitle{Motivation of the Lab}
 \begin{itemize}
 \item to present some of the data quality issues in real-world datasets.
 \item to explain how one can improve quality of such messy data.
 \item to offer a hands on practice for data cleaning.
 \end{itemize}
 \end{frame}
   
 \begin{frame}[c]
 \frametitle{A Real-World Scenario: The rodents dataset}
 \begin{itemize}
 \item data on rodents during a survey (over 25 years).
 \item each row denotes the information collected on an individual rodent.
 \item useful for studying population dynamics and species interactions.
 \item useful links:
 \begin{itemize}
 \item the original data is provided by Ernest et.\ al.\ \cite{Ernest2018}. 
 \item meta-data is also available at \footnotesize{\url{http://esapubs.org/archive/ecol/E090/118/Portal\_rodent\_metadata.htm}}.
  \end{itemize}
 \item Acknowledgement: this exercise is based on the Data Carpentry 
tutorial at
\footnotesize{\url{http://www.datacarpentry.org/OpenRefine-ecology-lesson/}}.
 \end{itemize}
 \end{frame}
 
 \begin{frame}[c]
  \frametitle{Data Quality}
 \begin{itemize}
 \item \textbf{inconsistencies}: 
 \begin{itemize}
 \item domain violation: a date entry of 4.31.2000, which is an invalid value (the intention might be 5.01.2000).
 \end{itemize}
 \item \textbf{missing data}: empty cells (some of them!), -99, etc. 
 \item \textbf{format variabilities} \footnote{several columns added for teaching purposes (see \footnotesize{\url{http://www.datacarpentry.org/OpenRefine-ecology-lesson/01-working-with-openrefine/}}).}: 
 \begin{itemize}
 \item typos: \say{Amph\textcolor{blue}{\textbf{e}}spiza bilineata} for \say{Amph\textcolor{blue}{\textbf{i}}spiza bilineata}.
 \item deduplication: \say{UNITED STATES} and \say{United States of America}.
 \item abbreviations: \say{US}. 
 \item leading and trailing whitespace: \say{  Amphispiza bilineata} and \say{  Amphispiza bilineata  }.
 \end{itemize}
  \end{itemize}
   \end{frame}

\begin{frame}[c]
\frametitle{Data Quality}
\begin{itemize}
 \item more issues for you to explore.
 \item repetitive tasks taking lots of time. 
 \item various tools that helps to transform such data: Trifacta \cite{Trifacta2018}, OpenRefine \cite{verborgh2013using}, etc.
\end{itemize}
\end{frame}

\begin{frame}[c]
\frametitle{Cleaning with OpenRefine}
\begin{itemize}
\item a tool for working with messy datasets.
\item see \cite{verborgh2013using} for details.
\item useful links: 
\begin{itemize}
\item the software at \footnotesize{\url{http://openrefine.org}}.
\item the documentation at \footnotesize{\url{https://github.com/OpenRefine/OpenRefine/wiki/Documentation-For-Users}}.
\end{itemize}
\item we will now install OpenRefine and describe its features for various data cleaning tasks.
\end{itemize}
\end{frame}

\begin{frame}[c]
\frametitle{Installation}
\begin{itemize}
\item \textbf{long answer}:
\begin{itemize}
\item detailed installation instructions at \footnotesize{\url{http://openrefine.org/download.html}}.
\end{itemize}
\item \textbf{short answer}:
\begin{itemize}
\item download the file depending on the OS at \footnotesize{\url{https://github.com/OpenRefine/OpenRefine/releases/tag/2.8}}.
\item install OpenRefine as follows:
\begin{itemize}
\item Linux: extract.
\item Mac: open, drag icon into the Applications folder.
\item Windows: unzip.
\end{itemize}
\end{itemize}
\end{itemize}
\end{frame}

\begin{frame}[c]
\frametitle{Running OpenRefine}
\begin{itemize}
\item \textbf{run OpenRefine} depending on the operating system:
\begin{itemize}
\item Linux: ./refine in your installation folder
\item Mac: OpenRefine in your Applications folder
\item Windows: .exe file in your installation folder
\end{itemize} 
\end{itemize}
\end{frame}

\begin{frame}[c]
\frametitle{Loading Data}
\begin{itemize}
\item \textbf{get the dataset}:
\begin{itemize}
\item clone the git repository at \footnotesize{\url{https://github.com/tahaceritli/acm-summer-school-2018}}.
\item use the file at datasets/Portal\_rodents\_19772002\_scinameUUIDs.csv.
\item the corresponding meta-data at meta-data/E090-118-D1-Rodent metadata.htm.
\end{itemize}
\item \textbf{import the data}:
\begin{itemize}
\item click \say{\textcolor{red}{Create Project}}.
\item click \say{\textcolor{red}{Choose Files}}.
\item select \textcolor{red}{Portal\_rodents\_19772002\_scinameUUIDs.csv}.
\item click \say{\textcolor{red}{Next}}.
\end{itemize} 
\end{itemize}
\end{frame}

\begin{frame}[c]
\frametitle{Data Preview}
\begin{itemize}
\item configuration page for importing.
\item a subset of the data is shown.
\item use the defaults.
\item click \say{\textcolor{red}{Create Project}} on the top-right corner.
\end{itemize}
\end{frame}

%\begin{frame}[c]
%\frametitle{Checking for Unique Values}
%\begin{itemize}
%\item click drop-down arrow in the \say{survey\_id} column.
%\item select \say{\textcolor{red}{Facet\textgreater Customized Facet\textgreater Duplicates Facet}}.
%\item results in a binary facet of \say{true} or \say{false}. 
%\item \say{true} facet denotes rows with unique values.
%\end{itemize}
%\end{frame}

\begin{frame}[c]
\frametitle{View Range of Values}
\begin{itemize}
\item click the drop-down arrow in the \say{scientificName} column.
\item select \say{\textcolor{red}{Facet\textgreater Text Facet}}.
\item lists the values and their counts.
\item any problems with the data?
\end{itemize}
\end{frame}

\begin{frame}[c]
\frametitle{Updating Cell Values}
\begin{itemize}
\item notice the spelling errors, e.g. \say{Amph\textcolor{blue}{\textbf{e}}spiza bilineata} for \say{Amph\textcolor{blue}{\textbf{i}}spiza bilineata}.
\item hover over the former and select \say{\textcolor{red}{edit}} to update its value.
\end{itemize}
\end{frame}

\begin{frame}[c]
\frametitle{Trimming Whitespace}
\begin{itemize}
\item leading and trailing whitespace: \say{  Amphispiza bilineata} and \say{  Amphispiza bilineata  }.
\item click the drop-down arrow in the \say{scientificName} column.
\item select \say{\textcolor{red}{Edit cells\textgreater Common transforms\textgreater Trim leading and trailing whitespace}}.
\end{itemize}
\end{frame}

\begin{frame}[c]
\frametitle{Clustering}
\begin{itemize}
\item data often contains more complex inconsistencies due to data collection procedures.
\item \say{Clustering} helps to find cells in a column, that refers to the same entity with different values.
\item various methods to determine clusters.
\end{itemize}
\end{frame}

\begin{frame}[c]
\frametitle{Clustering}
\begin{itemize}
\item click the drop-down arrow in the \say{scientificName} column.
\item select \say{\textcolor{red}{Edit cells\textgreater Cluster and edit...}}.
\item change the method to nearest neighbor.
\item you can now check boxes and merge the clusters.
\end{itemize}
\end{frame}

\begin{frame}[c]
\frametitle{Reconciliation}
\begin{itemize}
\item external knowledge to clean messy data.
\item reconciliation refers to the process of matching data to databases. 
\item popular knowledge bases: 
\begin{itemize}
\item Wikidata at \footnotesize{\url{https://www.wikidata.org/wiki/Wikidata:Main_Page}}. 
\item Google Knowledge Graph at \footnotesize{\url{https://www.google.com/intl/es419/insidesearch/features/search/knowledge.html}}.
\end{itemize}
\item OpenRefine uses Wikidata. details at \footnotesize{\url{https://github.com/OpenRefine/OpenRefine/wiki/Reconciliation}}.
\item note that Google Knowledge Graph comes with an API to query \footnotesize{\url{https://developers.google.com/knowledge-graph/}}.
\end{itemize}
\end{frame}

\begin{frame}[c]
\frametitle{Reconciliation with OpenRefine}
\begin{itemize}
\item click the drop-down arrow in the  \say{state} column.
\item select \say{\textcolor{red}{Reconcile\textgreater Start reconciling...}}.
\item select \say{\textcolor{red}{Wikidata}} and click \say{\textcolor{red}{Start Reconciling}}.
\item can take up to 1-2 minute to complete.
\item lists the matches with their confidence score.
\item two matching options: 
\begin{itemize}
\item with one tick to update the given cell only.
\item with two ticks to update all the cells with the same value.
\end{itemize}
\end{itemize}
\end{frame}

\begin{frame}[c]
\frametitle{Filtering Rows}
\begin{itemize}
\item click the drop-down arrow in the \say{scientificName} column.
\item select \say{\textcolor{red}{Text Filter}}.
\item type \say{bai}, which lists the first 10 matching rows.
\item searching rows based on multiple columns:
\begin{itemize}
\item click the drop-down arrow in the \say{scientificName} column.
\item select \say{\textcolor{red}{Facet\textgreater Custom text facet...}}.
\item using General Refine Expression Language (GREL).
\item and(cells[\say{scientificName}].value == \say{Amphispiza bilineata}, cells[\say{country}].value == \say{AUSTRALIA}).
\end{itemize}
\end{itemize}
\end{frame}

 \begin{frame}[c]
 \frametitle{Deliverables}
 \begin{enumerate}
% \item \textbf{unique values}: 
%\begin{itemize}
%\item report whether every ID in the column \say{survey\_id} is unique.
%\end{itemize}
 \item \textbf{inconsistencies}: 
 \begin{itemize}
\item report the number of rows that match 4, 31, 2000 in mo, dy, yr columns respectively.
\end{itemize}
\item \textbf{format variabilities}:
\begin{itemize}
\item report how many unique values exist in the \say{country} column of the original dataset.
\item are there any problems with the data? if so, explain how did you solve them? 
\item report how many unique values you have after fixing the potential inconsistencies.
\item hint: \say{clustering} or \say{reconciliation} feature of OpenRefine could be useful for this task.
 \end{itemize}
\newcounter{enumTemp}
\setcounter{enumTemp}{\theenumi}
 \end{enumerate}
\end{frame}

 \begin{frame}[c]
 \frametitle{Deliverables}
 \begin{enumerate}
 \setcounter{enumi}{\theenumTemp}
\item \textbf{missing data}:
\begin{itemize}
\item for the columns stake, species, county and nestdir:
\begin{itemize}
\item report the columns that have missing data and the encodings used to denote missing entries.
\item check whether they are actually missing (hint: you may want to check meta-data!).
\item hint: commonly used encodings: \say{NA}, \say{N/A}, \say{Null}, \say{-1}, \say{-99}, etc. 
\end{itemize}
%\item explain also if empty cells in notes (note1, note2, note3, note4, note5) denote missing data.
\end{itemize}
 \end{enumerate}
\end{frame}

\cut{
 \begin{frame}[c]
 \frametitle{Deliverables - Bonus}
 \begin{itemize}
 \item Traffic Violations dataset at \footnotesize{\url{https://catalog.data.gov/dataset/traffic-violations-56dda}}.
 \item import the subset of the original data given in datasets/traffic\_violations\_subset.csv.
 \item meta-data at meta-data/traffic\_violations.json.
 \item look at the columns named \say{Make} and \say{Driver City}.
 \item report the data quality issues.
 \item use OpenRefine to clean the data and explain which feature you used.
 \end{itemize}
\end{frame}
}

\begin{frame}[c]
 \frametitle{Submissions}
 \begin{itemize}
\item 1 page of either .txt or .pdf (not a .doc file!).
\item answer the four questions mentioned earlier.
\item send your report to \footnotesize{\url{acm2018datacleaninglab@gmail.com}} with the title ACMReport-NAME-SURNAME.
\item attach the exported file at step 2 to the email.
 \end{itemize}
\end{frame}

\begin{frame}[t, allowframebreaks]
\frametitle{References}
\bibliography{references}
\end{frame}

\end{document}

